
\chapter{RL - Indirizzamento degli operandi}
L'indirizzamento di memoria è complicato, ma fondamentale: il processore, per la maggior parte del tempo, copia pezzi di memoria da una parte a un'altra. L'indirizzamento di memoria è possibile sia con l'operando sorgente che con quello destinatario, ma non è possibile farlo in entrambi in una stessa istruzione (l'assemblatore da errore se ci proviamo). 
\paragraph{Caso più generale} Il caso più generico di indirizzo è il seguente
\[\text{Indirizzo}=\left|\text{base}+\text{indice}\times\text{scala} \pm \text{displacement}\right|_{2^{32}}\]
cioè 
\begin{verbatim}OPCODEsfx +-disp(base,indice,scala)\end{verbatim}
\begin{itemize}
	\item la base e l'indice consistono in registri generali a 32 bit. Precedentemente era obbligatorio porre un registro B in base e un registro I in indice: oggi si offre maggiore flessibilità e si possono utilizzare tutti i registri generali.
	\item scala è una costante che può avere per valore $1$ (valore default se non indicato), $2,4,8$.
	\item displacement è una costante intera. 
\end{itemize}
\subsection*{Indirizzamento di tipo diretto}
Nelle cose viste fino ad ora abbiamo fatto indirizzamenti di memoria diretti, cioè indirizzamenti con solo il displacement.
\begin{multicols}{2}
	\begin{verbatim}OPCODEW 0x00002001\end{verbatim}
	\[\text{Indirizzo}=\left| \text{0x00002001}\right|_{2^{32}}\]
\end{multicols}
\subsection*{Indirizzamento di tipo indiretto}
\subsubsection*{Registro base} 
\begin{multicols}{2}
	\begin{verbatim}OPCODEL (%EBX)
	\end{verbatim}
	\[\text{Indirizzo}=\left|\text{\%EBX}\right|_{2^{32}}\]
\end{multicols}
\noindent dove EBX consiste nel registro contenente l'indirizzo. Attenzione: è necessario indicare il suffisso di lunghezza, non abbiamo un indirizzamento di registri.
\clearpage 
\subsubsection*{Registro indice} 
\begin{multicols}{2}
	\begin{verbatim}OPCODEL (,%ESI, 4)\end{verbatim}
		\[\text{Indirizzo}=\left|\text{\%ESI}\times 4 \right|_{2^{32}}\]
	\end{multicols}
	\noindent dove ESI consiste nel registro contenente l'indirizzo. Attenzione: è necessario indicare il suffisso di lunghezza, non abbiamo un indirizzamento di registri.
	\subsubsection*{Indirizzamento con displacement e registro di modifica}
	\begin{multicols}{2}
		\begin{verbatim}OPCODEW 0x002A3A2B (%EDI)\end{verbatim}
			\[\text{Indirizzo}=\left|\text{\%EDI}+ \text{0x002A3A2B}\right|_{2^{32}}\]
		\end{multicols}
		\noindent Indirizzo un operando a 16bit, che si trova nella doppia locazione il cui indirizzo si ottiene sommando (modulo $2^{32}$) il displacement e il contenuto di EDI. Questa cosa è molto versatile per i vettori
		\subsubsection*{Indirizzamento bimodificato senza displacement}
		\begin{multicols}{2}
			\begin{verbatim}OPCODEW (%EBX, %EDI)\end{verbatim}
				\[\text{Indirizzo}=\left|\text{\%EBX}+\text{\%EDI}\times 1 \right|_{2^{32}}\]
			\end{multicols}
			\begin{multicols}{2}
				\begin{verbatim}OPCODEW (%EBX, %EDI, 8)\end{verbatim}
					\[\text{Indirizzo}=\left|\text{\%EBX}+\text{\%EDI}\times 8 \right|_{2^{32}}\]
				\end{multicols}
				\noindent utilizzo due registri puntatori ponendo, eventualmente, la scala.
				
				\subsubsection*{Indirizzamento bimodificato con displacement}
				In questo caso utilizzeremo tutte le armi a nostra disposizione
				\begin{multicols}{2}
					\begin{verbatim}OPCODEB 0x002F9000 (%EBX, %EDI)\end{verbatim}
						\[\text{Indirizzo}=\left|\text{\%EBX}+\text{\%EDI}\times 1 + \text{0x002F9000}\right|_{2^{32}}\]
					\end{multicols}
					\begin{multicols}{2}
						\begin{verbatim}OPCODEB -0x9000 (%EBX, %EDI)\end{verbatim}
							\[\text{Indirizzo}=\left|\text{\%EBX}+\text{\%EDI}\times 1 -\text{0x9000} \right|_{2^{32}}\]
						\end{multicols}