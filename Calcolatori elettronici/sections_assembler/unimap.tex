\chapter{Unimap} 
\begin{enumerate}
	\item \textbf{Lun 01/03/2021 10:30-12:30 (2:0 h)} lezione: Introduzione al corso. L'esempio dell'elaboratore SSEM: processore, memoria, linguaggio macchina, caricamento di un programma. (GIUSEPPE LETTIERI)
	\item \textbf{Mar 02/03/2021 10:30-12:30 (2:0 h)} lezione: Il processore Intel/AMD a 64 bit. Registri, modalità di accesso degli operandi, indirizzamento della memoria. (GIUSEPPE LETTIERI)
	\item \textbf{Gio 04/03/2021 14:30-16:30 (2:0 h)} lezione: Indirizzi canonici. Esempio di programma assembler: le direttive, le etichette. Utilizzo di assemblatore, collegatore e debugger. (GIUSEPPE LETTIERI)
	\item \textbf{Ven 05/03/2021 10:30-12:30 (2:0 h)} lezione: Spazio di indirizzamento: confini, offset, intervalli, allineamenti. Scomposizione degli indirizzi. (GIUSEPPE LETTIERI)
	\item \textbf{Lun 08/03/2021 10:30-12:30 (2:0 h)} lezione: Endianness. Parallelismo negli accessi allo spazio di memoria. Le linee di byte enable. Collegamento al bus di un modulo di RAM. (GIUSEPPE LETTIERI)
	\item \textbf{Mar 09/03/2021 10:30-12:30 (2:0 h)} esercitazione: Scrittura di programmi assembler su più file. (GIUSEPPE LETTIERI)
	\item \textbf{Gio 11/03/2021 14:30-16:30 (2:0 h)} lezione: Introduzione alla scrittura di programmi misti C++/Assembler. Funzione main. Registri preservati e scratch. Regole di allineamento dei tipi base e strutturati. (GIUSEPPE LETTIERI)
	\item \textbf{Ven 12/03/2021 10:30-12:30 (2:0 h)} lezione: Programmazione mista C++/Assembly: record di attivazione, passaggio dei parametri tramite i registri. (GIUSEPPE LETTIERI)
	\item \textbf{Lun 15/03/2021 10:30-12:30 (2:0 h)} esercitazione: Esercizi di traduzione da C++ ad Assembly, con passaggio di parametri per valore e riferimento. (GIUSEPPE LETTIERI)
	\item \textbf{Mar 16/03/2021 10:30-12:30 (2:0 h)} esercitazione: Esercizi su corrispondenza tra C++ e Assembly: funzioni con argomenti di tipo array e struttura. (GIUSEPPE LETTIERI)
	\item \textbf{Gio 18/03/2021 14:30-16:30 (2:0 h)} lezione: Memoria Cache: principi di località, controllore, memoria ad acesso diretto e associativa ad insiemi, politiche di write-through/write-back. (GIUSEPPE LETTIERI)
	\item \textbf{Ven 19/03/2021 10:30-12:30 (2:0 h)} lezione: Algoritmo pseudo-LRU per cache associative a 4 vie. Traduzione dei nomi C++ in assembler: funzioni globali con parametri di tipi base, definiti dall'utente e composti. (GIUSEPPE LETTIERI)
	\item \textbf{Lun 22/03/2021 10:30-12:30 (2:0 h)} lezione: Corrispondenza tra C++ e Assembler: classi, oggetti, funzioni membro, costruttori. (GIUSEPPE LETTIERI)
	\item \textbf{Mar 23/03/2021 10:30-12:30 (2:0 h)} lezione: Funzioni che restituiscono strutture, classi o unioni per valore. Regole per l'elisione dei costruttori di copia. Return Value Optimization e Named Return Value Optimization. (GIUSEPPE LETTIERI)
	\item \textbf{Gio 25/03/2021 14:30-16:30 (2:0 h)} esercitazione: Svolgimento di testi d'esame su corrispondenza tra C++ e Assembler. (GIUSEPPE LETTIERI)
	\item \textbf{Ven 26/03/2021 10:30-12:30 (2:0 h)} esercitazione: Svolgimento di testi d'esame su corrispondenza tra C++ e Assembler. (GIUSEPPE LETTIERI)
\end{enumerate}