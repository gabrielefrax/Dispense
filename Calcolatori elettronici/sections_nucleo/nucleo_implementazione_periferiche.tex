
\chapter{Codice del Modulo I/O}
\section{Contenuto della cartella \emph{io}}
Nella cartella \emph{io} troviamo i soliti file:
\begin{itemize}
	\item \emph{io.cpp},
	\item \emph{io.s}.
\end{itemize}
All'interno abbiamo l'implementazione di primitive e processi esterni per le periferiche che abbiamo già visto, in particolare tastiera, memoria video e hard disk. 
\paragraph{Timer} Il \emph{timer} è l'unica periferica implementata nel modulo sistema. Serve al sistema per implementare la \emph{delay}.
\section{Ritorniamo nella cartella \emph{include}}
In include abbiamo tre file contenenti le dichiarazioni delle primitive offerte. \subsection{\emph{sys.h}}
\emph{sys.h} contiene le primitive offerte dal modulo sistema, quelle che possono essere usate sia nel modulo utente che nel modulo sistema.
\small
\begin{verbatim}
	extern "C" natl activate_p(void f(natq), natq a, natl prio, natl liv);
	extern "C" void terminate_p();
	extern "C" natl sem_ini(int val);
	extern "C" void sem_wait(natl sem);
	extern "C" void sem_signal(natl sem);
	extern "C" void delay(natl n);
	
	extern "C" void do_log(log_sev sev, const char* buf, natl quanti);
	
	[...]
\end{verbatim}
\normalsize
\subsection{\emph{io.h}}
\emph{io.h} contiene le primitive implementate dal modulo I/O e rivolte al modulo utente.
\small
\begin{verbatim}
	extern "C" void iniconsole(natb cc);
	extern "C" natq readconsole(char* buff, natq quanti);
	extern "C" void writeconsole(const char* buff, natq quanti);
	
	extern "C" void readhd_n(void* vetti, natl primo, natb quanti);
	extern "C" void writehd_n(const void* vetto, natl primo, natb quanti);
	extern "C" void dmareadhd_n(void* vetti, natl primo, natb quanti);
	extern "C" void dmawritehd_n(const void* vetto, natl primo, natb quanti);
	
	[...]
\end{verbatim}
\normalsize
\begin{itemize}
	\item Le prime tre sono per tastiera e video.
	\item Le rimanenti sono per leggere e scrivere nell'hard disk. Si ha la versione in DMA e quella non in DMA per entrambe le operazioni (lettura / scrittura).
\end{itemize}
\subsection{\emph{sysio.h}}
\emph{sysio.h} contiene le primitive che il modulo sistema implementa \textbf{\underline{SOLO}} per il modulo I/O. 
\small
\begin{verbatim}
	extern "C" natl activate_pe(void f(int), int a, natl prio, natl liv, natb type);
	extern "C" void wfi();
	extern "C" void abort_p();
	extern "C" void io_panic();
	extern "C" paddr trasforma(void* ff);
	extern "C" bool access(const void* start, natq dim, bool writeable);
	extern "C" void fill_gate(natl tipo, vaddr f);
\end{verbatim}
\normalsize 
\begin{itemize}
	\item \textbf{activate$\_$pe} e \emph{wfi} (per le cose dette prima).
	\item \textbf{abort$\_$p}. Si permette di abortire i processi: la primitiva fa i controlli sui parametri passati dall'utente, se questi non vanno bene si deve abortire. \textbf{Idealmente solo il modulo sistema può abortire i processi}: per questo si fornisce una primitiva.
	\item \textbf{io$\_$panic} equivalente della \emph{panic} di sistema, se c'è qualche errore che impedisce di andare avanti.
	
	\item \textbf{fill$\_$gate}, chiamata dal modulo I/O per associare una propria funzione a una certa entrata della IDT (e quindi a un certo tipo).
	
	\item \textbf{access}: primitiva per le verifiche atte a risolvere il problema del cavallo di Troia (nel modulo sistema chiamiamo direttamente \emph{c$\_$access}, qua invochiamo la primitiva - negli esami facciamo la stessa cosa con la \emph{abort}).
	
	\item \textbf{trasforma}: primitiva per ottenere, dato un indirizzo fisico e un albero di traduzione, il corrispondente indirizzo virtuale.
\end{itemize}  

\section{File \emph{io.s}}
\subsection{Solite inizializzazioni} 
Come nel modulo sistema dobbiamo fare le solite inizializzazioni di costruttori di oggetti globali.
\small
\begin{verbatim}
	.text
	.global _start, start
	_start:
	start:
	// chiamiamo eventuali costruttori di oggetti globali
	movabs $__init_array_start, %rbx
	movabs $__init_array_end, %r12
	1:	 cmpq %r12, %rbx
	je 2f
	call *(%rbx)
	addq $8, %rbx
	jmp 1b
	// il resto dell'inizializzazione e' scritto in C++
	2:	 jmp main
\end{verbatim}
\normalsize 
Dopo aver fatto questo ci spostiamo nella funzione \emph{main}, implementata in C++ e contenente il resto delle inizializzazioni.
\subsection{Interfacce}
Il modulo presenta le interfacce per chiamare le primitive di sistema (\emph{chiamate di sistema generiche}), esattamente come il modulo sistema.
\begin{multicols}{2}
	\small 
	\begin{verbatim}
		.global getmeminfo
		getmeminfo:
		int $TIPO_GMI
		ret
		
		.global activate_p
		activate_p:
		int $TIPO_A
		ret
		
		.global terminate_p
		terminate_p:
		int $TIPO_T
		ret
		
		.global sem_ini
		sem_ini:
		int $TIPO_SI
		ret
		
		.global sem_wait
		sem_wait:
		int $TIPO_W
		ret
		
		.global sem_signal
		sem_signal:
		int $TIPO_S
		ret
		
		.global delay
		delay:
		int $TIPO_D
		ret
		
		.global do_log
		do_log:
		int $TIPO_L
		ret
	\end{verbatim}
	\normalsize 
\end{multicols}
\noindent Oltre a queste presenta le interfacce per le primitive da lui implementate (\emph{chiamate di sistema specifiche per il modulo I/O}).
\begin{multicols}{2}
	\small 
	\begin{verbatim}
		.global activate_pe
		activate_pe:
		int $TIPO_APE
		ret
		
		.global wfi
		wfi:
		int $TIPO_WFI
		ret
		
		.global abort_p
		abort_p:
		int $TIPO_AB
		ret
		
		
		
		
		
		.global io_panic
		io_panic:
		int $TIPO_IOP
		ret
		
		.global fill_gate
		fill_gate:
		int $TIPO_FG
		ret
		
		.global trasforma
		trasforma:
		int $TIPO_TRA
		ret
		
		.global access
		access:
		int $TIPO_ACC
		ret
	\end{verbatim}
	\normalsize 
\end{multicols}
\subsection{Gate per le primitive di I/O}
Come già anticipato la \emph{Interrupt  Descriptor Table} posta nel modulo sistema non è completa. Attraverso il seguente codice inizializziamo i gate rimanenti, necessari per l'esecuzione delle primitive.
\small 
\begin{verbatim}
	// Chiama fill_gate con i parametri specificati
	.macro fill_io_gate gate off
	movq $\gate, %rdi
	movabs $\off, %rax
	movq %rax, %rsi
	movq $LIV_UTENTE, %rdx
	call fill_gate
	.endm
	
	// Inizializzazione dei gate per le primitive di IO
	.global fill_io_gates
	fill_io_gates:
	pushq %rbp
	movq %rsp, %rbp
	
	fill_io_gate	IO_TIPO_RCON	a_readconsole
	fill_io_gate	IO_TIPO_WCON	a_writeconsole
	fill_io_gate	IO_TIPO_INIC	a_iniconsole
	fill_io_gate	IO_TIPO_HDR	a_readhd_n
	fill_io_gate	IO_TIPO_HDW	a_writehd_n
	fill_io_gate	IO_TIPO_DMAHDR	a_dmareadhd_n
	fill_io_gate	IO_TIPO_DMAHDW	a_dmawritehd_n
	fill_io_gate	IO_TIPO_GMI	a_getiomeminfo
	
	leave
	ret
\end{verbatim}
\normalsize 
Chiamo la \emph{fill$\_$gate} e associo i tipi delle primitive alle funzioni di ingresso \emph{a}. 

\subsection{Primitive di I/O}
Concludiamo ponendo la parte Assembler delle primitive implementate nel modulo I/O:
\begin{multicols}{2}
	\small
	\begin{verbatim}
		.extern c_readconsole
		a_readconsole:
		call c_readconsole
		iretq
		
		.extern c_writeconsole
		a_writeconsole:
		call c_writeconsole
		iretq
		
		.extern c_iniconsole
		a_iniconsole:
		call c_iniconsole
		iretq
		
		.extern	c_readhd_n
		a_readhd_n:
		call c_readhd_n
		iretq
		
		.extern	c_writehd_n
		a_writehd_n:
		call c_writehd_n
		iretq
		
		.extern	c_dmareadhd_n
		a_dmareadhd_n:
		call c_dmareadhd_n
		iretq
		
		.extern	c_dmawritehd_n
		a_dmawritehd_n:
		call c_dmawritehd_n
		iretq
		
		.extern	c_getiomeminfo
		a_getiomeminfo:
		call c_getiomeminfo
		iretq
	\end{verbatim}
	\normalsize 
\end{multicols}
\noindent Si osservi che non sono presenti la \emph{carica$\_$stato} e la \emph{salva$\_$stato}: 
\begin{itemize}
	\item non dobbiamo usarle perchè le primitive non sono atomiche;
	\item non possiamo usarle neanche per sbaglio, visto che le due funzioni non sono definite nel modulo I/O.
\end{itemize}
Si ha un semplice innalzamento di privilegio, al loro interno possono sospendersi e riprendersi utilizzando primitive di sistema.

\section{File \emph{io.cpp}}
\subsection{Funzione \emph{main}}
La funzione \emph{main} viene chiamata dalla \emph{start} vista in \emph{io.s}:
\small
\begin{verbatim}
	extern "C" void fill_io_gates();
	// eseguita in fase di inizializzazione
	extern "C" void main(int sem_io) {	
		fill_io_gates();
		mem_mutex = sem_ini(1);
		if (mem_mutex == 0xFFFFFFFF) {
			panic("impossible creare semaforo mem_mutex");
		}
		vaddr end_ = reinterpret_cast<vaddr>(&end);
		end_ = (end_ + DIM_PAGINA - 1) & ~(DIM_PAGINA - 1);
		heap_init((void *)end_, DIM_IO_HEAP);
		if (!console_init())
		panic("inizializzazione console fallita");
		if (!hd_init())
		panic("inizializzazione hard disk fallita");
		sem_signal(sem_io);
		terminate_p();
	}
\end{verbatim}
\normalsize 
\begin{itemize}
	\item Chiama \emph{fill$\_$io$\_$gates} per porre nella \emph{Interrupt Descriptor Table} i gate per il modulo I/O.
	\item Crea il semaforo \emph{mem$\_$mutex} con la primitiva \emph{sem$\_$ini}. Lo abbiamo trovato prima nelle funzioni per lo heap.
	\item Inizializza lo \emph{heap} con la funzione \emph{heap$\_$init}.
	\item Inizializza la console con \emph{console$\_$init}
	\item Inizializza l'hard disk con \emph{hd$\_$init}.
	\item La \emph{sem$\_$signal} è il rilascio di un gettone relativo alla sincronizzazione tra questo processo esterno e il processo sistema che sta inizializzando il modulo sistema. Abbiamo trovato la \emph{sem$\_$wait} nella funzione \emph{main$\_$sistema} del modulo sistema.
	\item Terminiamo il processo con \emph{terminate$\_$p}.
\end{itemize}


\subsection{Heap del modulo I/O}
Anche il modulo I/O ha il suo heap. Definiamo gli operator \emph{new} e \emph{delete}.
\small 
\begin{verbatim}
	natl mem_mutex;
	
	void* operator new(size_t s) {
		void *p;
		
		sem_wait(mem_mutex);
		p = alloca(s);
		sem_signal(mem_mutex);
		
		return p;
	}
	
	void operator delete(void* p) {
		sem_wait(mem_mutex);
		dealloca(p);
		sem_signal(mem_mutex);
	}
	
	[...]
\end{verbatim}
\normalsize 
La differenza rispetto al modulo sistema è il ricorso alle primitive semaforiche. Dobbiamo mantenere la mutua esclusione negli operatori (nel modulo sistema non abbiamo questo problema perchè lì le primitive sono atomiche per definizione).  