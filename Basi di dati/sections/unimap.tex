
\part{Unimap}
\small
\begin{itemize}
	\item \textbf{Mer 04/03/2020 08:30-11:30 (3:0 h)} lezione: Introduzione. Modello relazionale. (GIGLIOLA VAGLINI)
	\item \textbf{Gio 05/03/2020 10:30-12:30 (2:0 h)} non tenuta: Sospensione della didattica (GIGLIOLA VAGLINI)
	\item \textbf{Ven 06/03/2020 14:30-16:30 (2:0 h)} non tenuta: Sospensione della didattica (FRANCESCO PISTOLESI)
	\item \textbf{Mer 11/03/2020 08:30-11:30 (3:0 h)} lezione: Algebra relazionale. (GIGLIOLA VAGLINI)
	\item \textbf{Gio 12/03/2020 10:30-12:30 (2:0 h)} lezione: Introduzione e modalità d'esame. Il DBMS MySQL. Sintassi di una query: il SELECT statement. Processing. Condizioni e connettivi logici. Duplicati e keyword DISTINCT. Il valore NULL. Condizioni sui valori NULL. Gestione e formattazione di date. Funzioni DATEDIFF, PERIOD$\_$DIFF e DATE$\_$FORMAT. Lassi di tempo: la keyword INTERVAL. Shift temporale con DATE$\_$ADD/DATE$\_$SUB, e somma diretta. Condizioni temporali legate all'istante di esecuzione: l'impiego di CURRENT$\_$DATE. (FRANCESCO PISTOLESI)
	\item \textbf{Ven 13/03/2020 14:30-16:30 (2:0 h)} lezione: Funzioni di aggregazione: count, count(distinct), sum, avg, min e max. Ridenominazione. Il problema del record connesso a un valore aggregato. Multi-table querying. Inner join. Processing di una query con inner join. Query con join e condizioni sui record. Alias. Natural join. Cross join. Outer join. (FRANCESCO PISTOLESI)
	\item \textbf{Mer 18/03/2020 08:30-11:30 (3:0 h)} lezione: Calcolo relazionale (GIGLIOLA VAGLINI)
	\item \textbf{Gio 19/03/2020 10:30-12:30 (2:0 h)} esercitazione: Risoluzione ragionata degli esercizi per casa. Formulazione delle condizioni, connettivi logici, condizioni temporali e gestione delle date. Commento al codice. (FRANCESCO PISTOLESI)
	\item \textbf{Ven 20/03/2020 11:30-13:30 (2:0 h)} esercitazione: Progetto di interrogazioni con espressioni algebriche o nel calcolo relazionale (GIGLIOLA VAGLINI)
	\item \textbf{Ven 20/03/2020 14:30-16:30 (2:0 h)} lezione: Self join. Uso degli alias. Join multipli. Derived table. Subquery noncorrelated e correlated. Direttiva IN. Subquery scalari. Processazione in MySQL. Risoluzione mista subquery-join. Common Table Expressions (CTE). (FRANCESCO PISTOLESI)
	\item \textbf{Mer 25/03/2020 08:30-11:30 (3:0 h)} lezione: Progetto: modello concettuale (GIGLIOLA VAGLINI)
	\item \textbf{Gio 26/03/2020 10:30-12:30 (2:0 h)} esercitazione: Risoluzione ragionata degli esercizi per casa su join, subquery e funzioni di aggregazione. Esempio di traduzione passo-passo dalla versione con subquery alla versione join-equivalente. (FRANCESCO PISTOLESI)
	\item \textbf{Ven 27/03/2020 14:30-16:30 (2:0 h)} lezione: Raggruppamento. A cosa serve, come funziona e quando usarlo. La clausola GROUP BY. Condizioni sui gruppi: la HAVING clause. Processing in MySQL. Subquery EXISTS. Query con significato insiemistico. Unione di result set. Implementazione della divisione con doppia subquery NOT EXISTS, e con raggruppamento e subquery di conteggio nella having clause. (FRANCESCO PISTOLESI)
	\item \textbf{Mer 01/04/2020 08:30-11:30 (3:0 h)} lezione: La parte DD di SQL. Ristrutturazione di uno schema ER prima della traduzione. (GIGLIOLA VAGLINI)
	\item \textbf{Gio 02/04/2020 10:30-12:30 (2:0 h)} esercitazione: Risoluzione ragionata degli esercizi su query con raggruppamento con join e subquery. Uso delle CTE. (FRANCESCO PISTOLESI)
	\item \textbf{Ven 03/04/2020 11:30-13:30 (2:0 h)} lezione: Traduzione nel modello logico (GIGLIOLA VAGLINI)
	\item \textbf{Ven 03/04/2020 14:30-16:30 (2:0 h)} lezione: Query complesse. Ruolo nell'analisi predittiva, nei modelli decisionali, nel CRM. Strategie risolutive. Modificatori ANY/ALL. Gestire gli ex aequo. Differenza insiemistica. Introduzione alle stored procedure. Sintassi di CREATE PROCEDURE. Gestione degli statement. La keyword DELIMITER. Visualizzazione di result set su standard output. Chiamata a stored procedure. (FRANCESCO PISTOLESI)
	\item \textbf{Mer 08/04/2020 08:30-11:30 (3:0 h)} esercitazione: Esercizi di progetto, di traduzione e di ridondanza. (GIGLIOLA VAGLINI)
	\item \textbf{Gio 16/04/2020 10:30-12:30 (2:0 h)} esercitazione: Esercitazione sulle query complesse. (FRANCESCO PISTOLESI)
	\item \textbf{Ven 17/04/2020 14:30-16:30 (2:0 h)} lezione: Variabili locali e variabili user-defined in una stored procedure. Assegnamento con set e con select into. Parametri di una stored procedure: in, out, inout. Istruzioni condizionali: if-elseif-else, case. Istruzioni iterative: while, repeat e loop. Istruzioni di salto: leave e iterate. Cursori. Sintassi del comando declare cursor. Handler. Tipologie exit e continue. Ciclo di fetch. Exception handling e comando signal. Errori sqlstate. Not found condition. Stored function. Sintassi del comando create function. Esempio di stored function. Esempio di stored procedure che usa una stored function per realizzare un ranking tramite temporary table. (FRANCESCO PISTOLESI)
	\item \textbf{Mer 22/04/2020 08:30-11:30 (3:0 h)} lezione: Dipendenze funzionali e forme normali delle relazioni. (GIGLIOLA VAGLINI)
	\item \textbf{Gio 23/04/2020 10:30-12:30 (2:0 h)} esercitazione: Esercizi su stored procedure e stored function. Svolgimento di due testi d'esame. (FRANCESCO PISTOLESI)
	\item \textbf{Ven 24/04/2020 14:30-16:30 (2:0 h)} lezione: Introduzione ai database attivi. Trigger. Sintassi dell'istruzione create trigger. Trigger after e before. Meccanismo di scatto. Keyword 'new'. Gestione in sync di un attributo ridondante mediante trigger after e before. Business rule e gestione mediante trigger before. Event. Generalità sull'aggiornamento deferred. Pregi e difetti. Sintassi del comando create event. Scheduling di un event. Recurring event. Significato della direttiva on schedule at/every. Introduzione alle materialized view. Utilità nel reporting e nel data analytics. Performance. Politiche di refresh: immediate, deferred e on demand. (FRANCESCO PISTOLESI)
	\item \textbf{Mer 29/04/2020 08:30-11:30 (3:0 h)} lezione: Normalizzazione delle relazioni. (GIGLIOLA VAGLINI)
	\item \textbf{Gio 30/04/2020 10:30-12:30 (2:0 h)} esercitazione: Esercitazione su forme normali e normalizzazione delle relazioni. (GIGLIOLA VAGLINI)
	\item \textbf{Mer 06/05/2020 08:30-11:30 (3:0 h)} lezione: Esecuzione delle transazioni: recovery manager del DBMS (GIGLIOLA VAGLINI)
	\item \textbf{Gio 07/05/2020 10:30-12:30 (2:0 h)} esercitazione: Esercitazione sulle procedure di restart delle transazioni a seguito di guasti soft e hard. (GIGLIOLA VAGLINI)
	\item \textbf{Ven 08/05/2020 14:30-16:30 (2:0 h)} lezione: Esempio di materialized view con implementazione passo-passo del sync e del full refresh in modalità on demand e deferred. Incremental refresh. Log table. Trigger di push. Progettazione di log table efficienti: overhead vs. occupazione di memoria. Implementazione dell'incremental refresh. Processing parziale e totale della log table. Trasferimento delle modifiche nella materialized view. Le modalità partial e complete. (FRANCESCO PISTOLESI)
	\item \textbf{Mer 13/05/2020 08:30-11:30 (3:0 h)} lezione: Controllo della concorrenza nell'esecuzione delle transazioni: il concetto di scheduler (GIGLIOLA VAGLINI)
	\item \textbf{Gio 14/05/2020 10:30-12:30 (2:0 h)} esercitazione: Esercizi sulla serializzabilità degli schedule. (GIGLIOLA VAGLINI)
	\item \textbf{Ven 15/05/2020 11:30-13:30 (2:0 h)} esercitazione: Two-phase locking e time-stamp, differenze. (GIGLIOLA VAGLINI)
	\item \textbf{Ven 15/05/2020 14:30-16:30 (2:0 h)} esercitazione: Esercitazione su materialized view. Implementazione del full refresh mediante stored procedure ed event. Impostazione dell'esercizio per l'implementazione dell'incremental refresh e analisi preliminare della struttura della log table. (FRANCESCO PISTOLESI)
	\item \textbf{Mer 20/05/2020 08:30-11:30 (3:0 h)} lezione: Lo schema fisico delle basi di dati: indici, metodi di esecuzione degli operatori algebrici, in particolare i metodi per il join. Accesso, scansione e ordinamento. Calcolo delle dimensioni dei risultati intermedi. (GIGLIOLA VAGLINI)
	\item \textbf{Gio 21/05/2020 10:30-12:30 (2:0 h)} lezione: Calcolo delle dimensioni dei risultati intermedi: dettagli e esempi per i vari operatori. (GIGLIOLA VAGLINI)
	\item \textbf{Ven 22/05/2020 14:30-16:30 (2:0 h)} lezione: Introduzione alle window functions (analytic functions). Aggregate vs. non-aggregate functions. Clausola over. Definizione della partition e clausola partition by. Non-aggregate functions. Sort della partition. Uso combinato di partition by e order by. Le funzioni row$\_$number, rank, dense$\_$rank. Rank multipli. Funzioni lead e lag. Analisi delle frequenze relative cumulate tramite cume$\_$dist. Window functions su frame. Funzioni first$\_$value e last$\_$value. Definizione di frame mediante rows e range. Funzione moving average. Introduzione alle pivot table. Flat data. Operazione di pivoting. Pivoting statico. SQL dinamico. Prepared statement. Comandi prepare ed execute. Pivoting in SQL dinamico. (FRANCESCO PISTOLESI)
	\item \textbf{Mer 27/05/2020 08:30-11:30 (3:0 h)} esercitazione: Esercizi da esame da provare: algebra, calcolo, dipendenze funzionali e normalizzazione. (GIGLIOLA VAGLINI)
	\item \textbf{Gio 28/05/2020 10:30-12:30 (2:0 h)} esercitazione: Prove di esercizi d'esame: affidabilità, concorrenza e schema fisico. (GIGLIOLA VAGLINI)
	\item \textbf{Ven 29/05/2020 11:30-13:30 (2:0 h)} esercitazione: Simulazione di test di accesso all'orale. (FRANCESCO PISTOLESI,GIGLIOLA VAGLINI)
	\item \textbf{Ven 29/05/2020 14:30-16:30 (2:0 h)} esercitazione: Esercitazione su incremental refresh di materialized view. Tecniche di aggiornamento di attributi aggregati in modalità incremental. Esempio di gestione della media e dei valori massimi tramite concatenazione e parsing. Risoluzione ragionata e commento al codice. (FRANCESCO PISTOLESI)
	\item \textbf{Mer 03/06/2020 14:00-16:00 (2:0 h)} esercitazione: [Recupero del 06/03/2020] Esercizi e tutoring su window functions, tabelle pivot ed SQL dinamico. (FRANCESCO PISTOLESI)
\end{itemize}
\normalsize