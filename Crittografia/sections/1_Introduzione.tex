\chapter{Alcuni dettagli sul corso}
\paragraph{Obiettivo del corso e argomenti}
\[\boxed{\text{L’obiettivo del corso è comprendere le nozioni moderne dei cifrari moderni.}}\]
\begin{itemize}
	\item Faremo ciò da un punto di vista algoritmico, affrontando prima i cifrari storici . Sono richieste nozioni di algoritmica, ma anche di complessità computazionale. 
	\item Vedremo che alla base di un qualunque meccanismo di cifratura serve una chiave, e per tale motivo entreremo nel concetto di casualità e di numeri casuali: la casualità è necessaria per avere buone chiavi.
	
	\item Vedremo cifrari perfetti, che ci proteggono in modo completo se usati in modo corretto (soluzione ideale, anche se con un costo molto elevato facendo riferimento alla generazione delle chiavi).
	
	\item Vedremo cifrari simmetrici, cifratura a chiave pubblica di prima generazione (RSA) e seconda generazione (crittografia su curve ellittiche).
	
	\item Affronteremo anche il protocollo di autenticazione e le firme digitali, soffermandoci brevemente sul protocollo SSL.
	
	\item Concluderemo con argomenti più avanzati:
	\begin{itemize}
		\item protocollo “Zero Knowledge”;
		\item	elementi su valute virtuali e blockchain;
		\item	elementi di crittografia quantistica (protocollo per lo scambio delle chiavi, basato sui principi della meccanica quantistica).
	\end{itemize}
	Le nozioni relative all’ultimo punto non hanno nulla a che vedere col computer quantistico.
\end{itemize}
\paragraph{Libro suggerito}
Per seguire il corso è suggerito il libro 
\[\boxed{\text{Bernasconi, Ferragina, Luccio, Elementi di crittografia, Pisa Università Press 2015}}\]
I diritti d’autore, molto bassi, sono devoluti all’associazione \textit{Medici senza frontiere}.

\part{Introduzione}
\chapter{Introduzione alla crittologia}
\section{Crittografia e crittoanalisi}
Il termine crittografia significa “scrittura nascosta”. Distinguiamo due tipologie di studi:
\begin{itemize}
	\item \textbf{Crittografia} (metodi di cifratura).\\
	Studio di tecniche matematiche per mascherare messaggi, rendendoli incomprensibili a chi non è il legittimo destinatario.
	\item \textbf{Crittoanalisi} (metodi di interpretazione).\\
	Studio di tecniche matematiche per svelare messaggi criptati (si forza, si cerca di recuperare informazioni nascoste quando non si è il legittimo proprietario).
\end{itemize}
\noindent L’unione dei due termini da origine a un termine poco utilizzato: \textbf{crittologia}, lo studio della comunicazione su canali non sicuri e relativi problemi. La concezione del termine crittografia non è stretta: con questo termine facciamo molto spesso riferimento anche a questioni di crittoanalisi. 

%La crittografia (scrittura nascosta) è lo studio \emph{delle tecniche matematiche per mascherare i messaggi} a differenza della \emph{crittoanalisi} che tenta di svelarli.
%Esiste un termine più generico che li comprende: \emph{crittologia}.
\section{Scenario tipico della crittografia}
Il tipico scenario in cui ci poniamo è quello in cui Alice e Bob vogliono comunicare un messaggio \emph{m} su un canale insicuro dove è possibile intercettare i messaggi. Decidono quindi di adottare un \textbf{metodo di cifratura} che trasforma m in \emph{c} (\emph{c} è detto \textbf{crittogramma}), che deve essere:
\begin{itemize}
    \item \textbf{incomprensibile} al crittoanalista (Eve - Eavesdropper d'ora in poi)
    \item \textbf{facilmente decifrabile} da Bob
\end{itemize}

\subsection{Funzione di cifratura (criptaggio)}
L'operazione con la quale si trasforma m in c è di fatto una funzione:
$$
    C: \text{msg} \longrightarrow \text{critto}
$$
La funzione è applicata da chi spedisce il messaggio. Si passa dallo spazio dei messaggi allo spazio dei crittogrammi: chiaramente a un messaggio deve essere associato uno e un solo crittogramma (ricordare la definizione di funzione matematica\footnote{Da Wikipedia. In matematica, una funzione è una relazione tra due insiemi, chiamati dominio e codominio della funzione, che associa a ogni elemento del dominio uno e un solo elemento del codominio.}).
\subsection{Funzione di decifratura (decifrazione)}
L'operazione inversa:
$$
    D: \text{critto} \longrightarrow \text{msg}
$$
La funzione è applicata da chi riceve il messaggio e vuole decifrarlo. Si passa dallo spazio dei crittogrammi allo spazio dei messaggi. 
\subsection{Schema di comunicazione}
$$
    \text{Alice}: m \xrightarrow{C} c \xrightarrow[\text{canale insicuro}]{c} c \xrightarrow{D} m : \text{Bob}
$$
Si noti che per funzionare \emph{C} e \emph{D} devono essere in tempo polinomiale mentre per il crittoanalista, noto \emph{c}, deve essere esponenziale il tempo utile per riottenere \emph{m}.
\paragraph{NB} \emph{C} e \emph{D} devono essere l'una l'inversa dell'altra:
$$
    D(c) = D(C(m)) = m
$$
quindi \emph{C} è iniettiva: m diversi vanno in c diversi (altrimenti il destinatario Bob non potrà determinare in modo univoco il messaggio a lui rivolto).

\section{Esempi antichi}
\paragraph{Erodoto in "Storie" (V secolo a.C.)} Si prende un servitore, si rasano i suoi capelli e si scrive il messaggio sulla sua testa, si aspetta che la ricrescita lo copra e poi si spedisce il servitore verso Bob che dovrà solamente rasarlo nuovamente.

\paragraph{Spartani (V secolo a.C.)} Gli spartani (V secolo a.C.) usavano lo scitale che è un'asta cilindrica costruita in due esemplari identici posseduti dai due corrispondenti. Su un pezzo di pelle viene scritto il messaggio dopo averlo avvolto attorno al cilindro seguendo le linee di esso. La fettuccia viene poi fatta indossare da un'uomo che la porta al ricevente.

\paragraph{Enea Tattico (Grecia, IV Secolo a.C.)} Enea tattico dedica un intero capitolo ai metodi militari usati per scambiarsi i messaggi:
\begin{itemize}
    \item inviare un libro con alcune lettere sottolineate a formare il messaggio in chiaro
    \item sostituire le vocali con altri simboli
\end{itemize}

\paragraph{Cifrario di Cesare} Il Cifrario di Cesare è il più antico cifrario di concezione moderna. \emph{c} è ottenuto da \emph{m} sostituendo ogni lettera con quella a 3 posizioni più avanti:
\begin{align*}
	A &\longrightarrow D \\
	B &\longrightarrow E \\
	C &\longrightarrow F \\
	D &\longrightarrow G\\
	.. &\longrightarrow ..
\end{align*}
Nella decifrazione faremo la stessa cosa al contrario. La segretezza in questo caso dipende dalla conoscenza del metodo: se il metodo viene scoperto allora non può essere più utilizzato!

\section{Livello di segretezza}
I metodi crittografici si classificano in:
\begin{itemize}
    \item \textbf{per uso ristretto} (usati in ambito diplomatico e/o militare\footnote{Si tenga conto che maggiore è il numero di persone coinvolte, maggiore è il rischio che il segreto venga rivelato da persone che ne sono a conoscenza. }): in cui la parte segreta del meccaniscmo è ampia (C e D sono tenute segrete)
    \item \textbf{per uso generale} (rivolto alla crittografia di massa): in cui la parte segreta è molto limitata (si restringe alla sola \emph{chiave}, nota solo ai due endpoint della comunicazione)
\end{itemize}
NB: per i cifrari di massa quindi le regole sono pubblcihe, solo le chiavi sono segrete. \textbf{Occorre sempre pensare che il nemico conosca il sistema}. 

\paragraph{Proporzione} Maggiore è la dimensione della parte segreta del cifrario minore deve essere il numero delle persone a conoscenza della parte segreta.

\subsection{Ridefinizione delle funzioni di criptaggio e decifrazione}
Ridefiniamo quindi:
\begin{align*}
	c = C(m, k) && m = D(c, k)
\end{align*}
con \emph{k} chiave segreta diversa per ogni coppia di utenti.
Se non si conosce \emph{k} la conoscenza dell'algoritmo non deve permettere l'estrazione di informazioni dal crittogramma. Se una chiave viene divulgata basta generarne un'altra lasciando inalterati \emph{C} e \emph{D}.
Ovviamente l'insieme delle chiavi deve esere così grande da non essere rompibile tramite brute-force e deve essere scelta in modo casuale.
\[\text{NB: brute-force $\equiv$ \emph{attacco esauriente}}\]
\paragraph{Esempio} Se $|key| = 10^{20}$ ed un calcolatore impiegasse $10^{-6}$ secondi per calcolare $D(c, k)$ e verificarne la significatività occorrerebbero comunque milioni di anni per provarle tutte.\\NB: solo la grandezza dello spazio delle chiavi non è un buon indice per l'affidabilità di un cifrario, potrebbe sempre essere rotto matematicamente (la segretezza non deve dipendere esclusivamente dalla cardinalità dello spazio).

\section{Crittoanalisi}
Il crittoanalista può assumere uno dei seguenti comportamenti.
\begin{itemize}
    \item Comportamento \emph{passivo}: si limita ad ascoltare il canale
    \item Comportamento \emph{attivo}:  disturba le comunicazioni o modifica il contenuto dei messaggi
\end{itemize}
Gli attacchi hanno lo scopo di forzare il sistema, raggiungendo l’informazione protetta da quel particolare sistema. L'attacco al sistema crittografico può avere due esiti:
\begin{itemize}
	\item \textbf{successo}, si scopre la funzione D, o si trova la chiave (e quindi possiamo decifrare qualunque crittogramma);
	\item \textbf{successo limitato}, si scoprono informazioni parziali, ma sufficienti per comprendere il contenuto del messaggio.
\end{itemize}
\subsection{Tipologie di attacchi}
Gli attacchi dipendono dalle informazioni in possesso del crittoanalista:
\begin{itemize}
    \item \textbf{\emph{cipher text attack}} (solo testo cifrato): si hanno una serie di crittogrammi:
    \[c_{1}, \dots , c_{r}\]
    \item \textbf{\emph{known plain-text attack}} (testo in chiaro noto): il crittoanalista ha a disposizione delle coppie
    $$(m_{1}, c_{1}), \dots , (m_{r}, c_{r})$$
    i messaggi non sono scelti dal crittoanalista
    \item \textbf{\emph{chosen plain-text attack}} (testo in chiaro scelto): ci si procura una serie di coppie
    $$(m_{1}, c_{1}), \dots , (m_{r}, c_{r})$$
    relative a messaggi in chiaro scelti dal crittoanalista
\end{itemize}

\subsection{Attacchi man-in-the-middle}
Il crittoanalista si installa sul canale ed interrompe le comunicazioni dirette tra i due, le sostituisce con messaggi propri e convince ogni utente che quei messaggi provengono legittimamente dall'altro.
\[\text{\underline{Il crittoanalista si finge Alice agli occhi di Bob e viceversa.}}\]

%\section{Cifrari perfetti e definizione di cifrario inattaccabile}
%Si conoscono alcuni \textbf{cifrari perfetti} ma richiedono operazioni estremamente complesse, e quindi sono utilizzati in condizioni estreme. La definizione di \textbf{cifrario inattaccabile} si deve a \emph{Claude Shannon} ('45 ma pubblicato nel '49 per segreto di stato imposto dai britannici)
%\[\text{\underline{Il messaggio in chiaro ed il crittogramma sono completamente scorrelati tra loro}}\]
%\section{\emph{one-time pad}}Un esempio di cifrario sicuro è il \emph{one-time pad} che richiede:
%\begin{itemize}
%    \item una chiave diversa per ogni messaggio
%    \item perfettamente casuale
%    \item lunga quanto il messaggio
%\end{itemize}
%Come vanno generate? Come vanno scambiate? Emergono problematiche su:
%\begin{itemize}
%	\item dimensione del messaggio e della chiave;
%	\item metodologie per far scambiare le chiavi tra i due endpoint.
%\end{itemize}
%\section{Cifrari dichiarati sicuri}I cifrari utilizzati oggigiorno non sono perfetti ma sono comunque \textbf{dichiarati sicuri} perché inviolati e per violarli è necessario risolvere problemi matematici estremamente difficili (abbiamo solo algoritmi esponenziali), quindi è richiesto tanto tempo o calcolatori molto grandi: nella pratica la risoluzione è impossibile.

%\paragraph{NB} Non sempre è noto se l'algoritmo esponenziale è l'unico metodo o ce ne sono altri ancora non scoperti.

%\section{\emph{Advanced Encryption Standard} (AES)}
%Uno dei cifrari di oggi è l'\emph{AES} (Advanced Encryption Standard): è lo standard per le comunicazioni non classificate, pubblicamente noto ed implementabile. Usa chiavi brevi a 128 o 256 bit. E' un cifrario simmetrico a blocchi, cioè si usa la stessa chiave sia per cifrare che per decifrare.

%\paragraph{Novità} La chiave non è scelta dai partecipanti ma dai dispositivi elettronici utilizzati. Ad ogni sessione viene utilizzata una nuova chiave (la chiave non deve durare a lungo). 

%\paragraph{Domanda} Come trasmettere la chiave in maniera sicura evitando intercettazioni?

%\section{Distribuzione delle chiavi su canali insicuri}
%Nel 1976 è stato proposto un protocollo di creazione e scambio di chiavi su un canale insicuro senza la necessità che le due parti debbano essersi scambiate altre informazioni. Questo algoritmo è detto \emph{protocollo Diffie-Hellman} ed è tuttora usato largamente.

%\paragraph{NB} Inventato da Merkle e poi da Diffie ed Hellman. Questi stessi hanno anche creato il concetto di crittografia a chiave pubblica senza tuttavia fornire un'implementazione.

%\section{Cifrari simmetrici ed asimmetrici}
%Nei cifrari simmetrici la chiave è unica e usata sia per criptare che per decriptare, ed è nota solo ai due partner che devono averla concordata su un canale sicuro. Nei cifrari asimmetrici, invece, si usa una coppia di chiavi:
%\begin{itemize}
%    \item \emph{$K_{pub}$}: è pubblica e nota a tutti (un po' come un lucchetto, tutti possono chiuderlo);
%    \item \emph{$K_{priv}$}: è privata e nota solo a chi riceve.
%\end{itemize}
%Bisogna quindi creare delle coppie di chiavi per ogni persona che vuole comunicare. Più precisamente:
%\begin{align*}c = C(m, K_{pub})&&m = D(c, K_{priv})\end{align*}
%I sistemi simmetrici si dicono anche a chiave privata, mentre quelli asimmetrici si dicono a chiave pubblica.
%Per usare un meccanismo come crittografia asimmetrica è necessario l'uso di funzioni \emph{one-way trapdoor} cioè passare da \emph{m} a \emph{c} è facile ma decifrare \emph{c} (senza conoscere la chiave) è difficile.

%\section{RSA (Rivest–Shamir–Adleman)}
%Nel 1977 Rivest-Shamir-Adleman inventano un sistema a chiave pubblica basato sulla difficoltà nel fattorizzare grandi numeri in fattori primi.
%Usando un sistema a chiave pubblica si ha una comunicazione molti a uno in quanto tutti hanno $K_{pub}$ e possono quindi cifrare i messaggi, ma solo il destinatario può leggerli conoscendo $K_{priv}$. Altri vantaggi sono:
%\begin{itemize}
%    \item tra n persone le chiavi sono $2n$, con un cifrario simmetrico invece sarebbero $\frac{n(n-1)}{2}$
%    \item non è necessario lo scambio segreto di chiavi
%\end{itemize}
%Tuttavia sono molto più lenti e le chiavi sono molto lunghe. Sono soggetti ad attacchi di tipo \emph{chosen plain-text attack}.

%\section{Cifrari ibridi}
%Si usa un cifrario a chiave segreta per la comunicazione di massa ma la chiave viene scambiata tramite cifrari asimmetrici. Si hanno quindi:
%\begin{itemize}
%    \item chiavi piccole
%    \item chiave simmetrica randomica impossibile da prevedere tramite attacco di tipo testo in chiaro scelto
%\end{itemize}

%\section{Applicazioni moderne}
%Attualmente i protocolli crittografici sono usati per:
%\begin{itemize}
%    \item \emph{identificazione dell'utente}: si accerta l'identità
%    \item \emph{autenticazione dell'utente}: si accerta che il messaggio venga dalla persona che dice di averlo mandato
%    \item \emph{firma digitale}: permette di evitare che un utente che ha inviato un messaggio neghi di averlo fatto e si dimostra l'identità del mittente agli occhi del ricevente
%\end{itemize}

%\section{Svolte future}
%\begin{itemize}
%    \item trasmissione protetta sulla rete (OpenSSL)
%    \item moneta elettronica (protocollo Bitcoin)
%    \item protocolli zero-knowledge
%    \item protocolli di cifratura quantistici
%\end{itemize}